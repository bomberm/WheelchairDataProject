\documentclass[onecolumn, draftclsnofoot,10pt, compsoc]{IEEEtran}

%slightly modified from stackoverflow @ https://tex.stackexchange.com/questions/200437/numbering-sections-subsections-etc-manually
%code block below allows for references to function as a section instead of a chapter
\makeatletter
\renewenvironment{thebibliography}[1]
{\subsection{References}
	\@mkboth{\MakeUppercase\bibname}{\MakeUppercase\bibname}%
	\list{\@biblabel{\@arabic\c@enumiv}}%
	{\settowidth\labelwidth{\@biblabel{#1}}%
		\leftmargin\labelwidth
		\advance\leftmargin\labelsep
		\@openbib@code
		\usecounter{enumiv}%
		\let\p@enumiv\@empty
		\renewcommand\theenumiv{\@arabic\c@enumiv}}%
%	\sloppy
	\clubpenalty4000
	\@clubpenalty \clubpenalty
	\widowpenalty4000%
	\sfcode`\.\@m}
{\def\@noitemerr
	{\@latex@warning{Empty `thebibliography' environment}}%
	\endlist}
\makeatother

\usepackage{graphicx}
\usepackage{url}
\usepackage{setspace}
\makeindex
\usepackage{geometry}

\geometry{textheight=9.5in, textwidth=7in}

% 1. Fill in these details
\def \CapstoneTeamName{		Wheelchair Data Collection Team}
\def \CapstoneTeamNumber{		4}
\def \GroupMemberOne{			Marie Bomber}
\def \GroupMemberTwo{			Aaron Leondar}
\def \GroupMemberThree{			Hadi Rahal-Arabi}
\def \CapstoneProjectName{			Robotic Wheelchair Data Collection and Analysis}
\def \CapstoneSponsorCompany{	Oregon State University}
\def \CapstoneSponsorPerson{	Matthew William Shuman	}

% 2. Uncomment the appropriate line below so that the document type works
\def \DocType{	%Problem Statement
				%Requirements Document
				%Technology Review
				Design Document
				%Progress Report
				}
\bibliographystyle{ieeetran}	
\newcommand{\NameSigPair}[1]{\par
\makebox[2.75in][r]{#1} \hfil 	\makebox[3.25in]{\makebox[2.25in]{\hrulefill} \hfill		\makebox[.75in]{\hrulefill}}
\par\vspace{-12pt} \textit{\tiny\noindent
\makebox[2.75in]{} \hfil		\makebox[3.25in]{\makebox[2.25in][r]{Signature} \hfill	\makebox[.75in][r]{Date}}}}
% 3. If the document is not to be signed, uncomment the RENEWcommand below
%\renewcommand{\NameSigPair}[1]{#1}

%%%%%%%%%%%%%%%%%%%%%%%%%%%%%%%%%%%%%%%
\begin{document}
\begin{titlepage}
    \pagenumbering{gobble}
    \begin{singlespace}
        \hfill 
        % 4. If you have a logo, use this includegraphics command to put it on the coversheet.
        %\includegraphics[height=4cm]{CompanyLogo}   
        \par\vspace{.2in}
        \centering
        \scshape{
            \huge CS Capstone \DocType \par
            {\large 27 October 2017}\par
            \vspace{.5in}
            \textbf{\Huge\CapstoneProjectName}\par
            \vfill
            {\large Prepared for}\par
            \Huge \CapstoneSponsorCompany\par
            \vspace{5pt}
            {\Large\NameSigPair{\CapstoneSponsorPerson}\par}
            {\large Prepared by }\par
            Group\CapstoneTeamNumber\par
            % 5. comment out the line below this one if you do not wish to name your team
            \CapstoneTeamName\par 
            \vspace{5pt}
            {\Large
                \NameSigPair{\GroupMemberOne}\par
                \NameSigPair{\GroupMemberTwo}\par
                \NameSigPair{\GroupMemberThree}\par
            }
            \vspace{20pt}
				\begin{abstract}
This document serves to provide the overall design of the entire project. It includes the communication between the wheelchair computer and the personal computer, the methods used to store and present the data, and the interface that brings the entire process together. All of the technologies talked about in this document are the technologies that were chosen by our team to use with our project because they were determined to be the most advantageous to the goals of our project. This document does not contain the alternative technologies that our team looked at and researched, but ultimately did not choose.
				\end{abstract} 
        }   
    \end{singlespace}
\end{titlepage}
\newpage
\pagenumbering{arabic}
\tableofcontents
%\listoffigures
%\listoftables
\clearpage

% 8. now you write!
\section{Introduction}


\subsection{Glossary}
\begin{description}
	\item [Bag] \hfill \break File format used in ROS for storing ROS message data
	\item [Researcher] \hfill \break Admin level user of the Project Chiron Interface. The researcher is expected to have full access to all user testing data (including names)
	\item [POMDP] \hfill \break Partially Observable Markov Decision Process. Per POMDP.org "This is a mathematical model that can capture the domain dynamics that include uncertainty in action effects and uncertainty in perceptual stimuli. Once a problem is captured as POMDP, it them becomes more amendable for solution using optimization techniques."  \cite{POMDP}
	\item [ROS] \hfill \break Robot Operating System
	\item [Tester] \hfil \break Mid level user of the Project Chiron Interface. The tester is expected to be able to initiate a user test, but will not be able to retrieve all user data. (May be able to retrieve data via an ID, but will not have access to research participant names or prior test results).
	\item [Research Participant] \hfill \break Individual who has no access to the Project Chiron interface, but will have a name and ID entry and who will run a user test.
	\item [User Test] \hfill \break Instance of a single navigation of the testing course. This test will produce a bag of testee data that the Chiron interface must be able to store and retrieve.
\end{description}
\section{Organization and Project Management}
\subsection{Overview}
Because the project includes three distinct design layers that will need to interact with complex ways, we have decided to include a project manager role to manage system design and delegation of tasks. This section will focus on the technologies to aid in project management.
\subsection{Design Concerns}

\subsection{Design Elements}

\section{Communication Layer}

\subsection{Overview}
The Communication Layer represents the interaction between the controllable wheelchair and the computer that the researcher is handling.  ROS, other than being used to communicate between the wheelchair and the local computer, also will be used to gather data when the wheelchair is run. The data collected will be stored in files called bags. These bags will contain the time from start to finish that the wheelchair was run, as well as any relevant sensor data. After the wheelchair is run and the bag collected, it is then transferred from the raspberry pi to the local computer, and stored on the computer.

\subsection{Design Concerns}
The bags must contains only relevant data. By default, bags store quite a lot of data that is completely unnecessary to our project, so all unnecessary data must not be gathered by the bag. If superfluous data is gathered, it causes the bag filesize to increase, which ia a concern since dozens of bags must fit on a single small 8GB thumb drive. It is also highly important to have a stable wireless connection, or else data will not be able to be properly transferred between the robot and the computer, or will not be transferred at all.

\subsection{Design Elements}
The wheelchair contains a raspberry pi computer that has ROS installed on it, as well as a wireless router that allows a local computer with Wi-Fi capabilities to connect to it. The local computer will be any computer that can run Linux, as Linux is required to run ROS on the wheelchair, and is Wi-Fi enabled to be able to communicate with the wheelchair. The computer will have to start out connected to the internet in order to connect to the robot. The computer will also need to stay connected while the robot performs its trial. Then once the trial ends, the bag with all the data collected from the trial will be transferred to the computer and stored.

\section{Storage Layer}
\subsection{Overview}
Once the ROSbag is received and processed by the system, it needs to be stored. This test information and the associated metadata, such as research participants name, time and date of trial and test instance, must be kept in such a way that the tester cannot retrieve the information, but the researcher can. This section will focus on the design concerns and decisions related to ROSbag storage. 
\subsection{Design Concerns}
The storage layer must be able to keep all test participant information secure. The researcher (and therefore, the application) has two competing needs in this regard. First, they need the test participants name as part of the initial research, this allows them to assess the test participants competency using the robotic wheelchair. Second, once the research is complete, this link between the test participant and the test data must be able to be severed completely once the initial stage of the research is complete. In addition, the researcher will not be the individual initiating and running the participant tests, so the ROSbag information must be stored securely so the tester cannot view the test information once the test is complete, but this must not interfere with the researchers access. Lastly, our client has stated a preference for a locally run application which excludes a externally hosted server.
\subsection{Design Elements}
To meet these competing design goals, we have selected a serverless database solution. This gives the flexibility of a relational database, but without the overhead of running a database server, as serverless database formats contain all data in a single file. In addition, we will be able to encrypt test and test participant information and give the researcher the ability to remove the test participants name to their test data at the end of the first stage of research. 

\section{UI Layer}
  
\section{Conclusion}

\bibliography{DesignDocument}

\end{document}