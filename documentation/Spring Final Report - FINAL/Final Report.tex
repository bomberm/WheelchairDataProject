\documentclass[onecolumn, draftclsnofoot,10pt, compsoc]{report}

%slightly modified from stackoverflow @ https://tex.stackexchange.com/questions/200437/numbering-sections-subsections-etc-manually
%code block below allows for references to function as a section instead of a chapter
\makeatletter
\renewenvironment{thebibliography}[1]
{\subsection{References}
	\@mkboth{\MakeUppercase\bibname}{\MakeUppercase\bibname}%
	\list{\@biblabel{\@arabic\c@enumiv}}%
	{\settowidth\labelwidth{\@biblabel{#1}}%
		\leftmargin\labelwidth
		\advance\leftmargin\labelsep
		\@openbib@code
		\usecounter{enumiv}%
		\let\p@enumiv\@empty
		\renewcommand\theenumiv{\@arabic\c@enumiv}}%
%	\sloppy
	\clubpenalty4000
	\@clubpenalty \clubpenalty
	\widowpenalty4000%
	\sfcode`\.\@m}
{\def\@noitemerr
	{\@latex@warning{Empty `thebibliography' environment}}%
	\endlist}
\makeatother

\usepackage{graphicx}
\usepackage{url}
\usepackage{setspace}
\makeindex
\usepackage{geometry}
\usepackage{minitoc}
\usepackage{titlesec}

\titleformat{\chapter}[display]
{\normalfont\bfseries}{}{0pt}{\Huge}

\setcounter{minitocdepth}{1}

\geometry{textheight=9.5in, textwidth=7in}

% 1. Fill in these details
\def \CapstoneTeamName{		Wheelchair Data Collection Team}
\def \CapstoneTeamNumber{		4}
\def \GroupMemberOne{			Marie Bomber}
\def \GroupMemberTwo{			Aaron Leondar}
\def \GroupMemberThree{			Hadi Rahal-Arabi}
\def \CapstoneProjectName{			Robotic Wheelchair Data Collection and Analysis}
\def \CapstoneSponsorCompany{	Oregon State University}
\def \CapstoneSponsorPerson{	Matthew William Shuman	}

% 2. Uncomment the appropriate line below so that the document type works
\def \DocType{	%Problem Statement
				%Requirements Document
				%Technology Review
				%Design Document
					Final Report
				}
			
\renewenvironment{abstract}{%
	\hfill\begin{minipage}{0.95\textwidth}
		\rule{\textwidth}{1pt}}
	{\par\noindent\rule{\textwidth}{1pt}\end{minipage}}

\bibliographystyle{ieeetran}	
\newcommand{\NameSigPair}[1]{\par
\makebox[2.75in][r]{#1} \hfil 	\makebox[3.25in]{\makebox[2.25in]{\hrulefill} \hfill		\makebox[.75in]{\hrulefill}}
\par\vspace{-12pt} \textit{\tiny\noindent
\makebox[2.75in]{} \hfil		\makebox[3.25in]{\makebox[2.25in][r]{Signature} \hfill	\makebox[.75in][r]{Date}}}}
% 3. If the document is not to be signed, uncomment the RENEWcommand below
%\renewcommand{\NameSigPair}[1]{#1}

%%%%%%%%%%%%%%%%%%%%%%%%%%%%%%%%%%%%%%%
\begin{document}
\begin{titlepage}
    \pagenumbering{gobble}
    \begin{singlespace}
        \hfill 
        % 4. If you have a logo, use this includegraphics command to put it on the coversheet.
        %\includegraphics[height=4cm]{CompanyLogo}   
        \par\vspace{.2in}
        \centering
        \scshape{
            \huge CS Capstone \DocType \par
            {\large 12 June 2018}\par
            \vspace{.5in}
            \textbf{\Huge\CapstoneProjectName}\par
            \vfill
            {\large Prepared for}\par
            \Huge \CapstoneSponsorCompany\par
            \vspace{5pt}
            {\Large\NameSigPair{\CapstoneSponsorPerson}\par}
            {\large Prepared by }\par
            Group\CapstoneTeamNumber\par
            % 5. comment out the line below this one if you do not wish to name your team
            \CapstoneTeamName\par 
            \vspace{5pt}
            {\Large
                \NameSigPair{\GroupMemberOne}\par
                \NameSigPair{\GroupMemberTwo}\par
                \NameSigPair{\GroupMemberThree}\par
            }
            \vspace{20pt}
\begin{abstract}
% Add Abstract
				\end{abstract} 
        } 
    \end{singlespace}
\end{titlepage}
\newpage
\pagenumbering{arabic}
\dominitoc
\tableofcontents
%\listoffigures
%\listoftables
\clearpage


\chapter{Introduction to Project}
\minitoc

\chapter{Requirements Document}
\minitoc

\chapter{Design Document}
\minitoc

\chapter{Tech Review}
\minitoc

\chapter{Weekly Blog Posts}
\minitoc
\section{Marie's Posts}
\section{Aaron's Posts}
\section{Hadi's Posts}

\chapter{Final Poster}
\minitoc

\chapter{Project Documentation}

\minitoc
\section{Getting Started}


Clone the ROS Test Controller at https://github.com/htrarabi/ROSTestController into a directory on your robot. Ensure your robot has internet access before proceeding with installation.

\section{Prerequisites}

The ROS Test Controller assumes your robot is running Ubuntu 15.10 or higher and has been testing only with ROS Kinetic Kame. While it may work with later ROS distos, it has not been tested and you may get unexpected behavior.

In addition, your robot must have full internet access in order for installation to be successful. After installation, the controller will function without web access, but it's appearance may change slightly.

Because the ROS Test Controller uses both nodeJS and Python 2.7 to control it's behavior, the installer will install these programs if they are not already installed on your robot. In addition, both pip and npm will be installed alongside python and nodeJS.
\section{Installing}

To install the ROS Test Controller, simply run \texttt{sudo -H ./install.sh} in the top level of the ROSTestController directory. This script will confirm that Python 2.7, pip, nodeJS and npm are installed on your robot and install them if not. In addition, install.sh will install several libraries that the Contoller depends on. Once complete, the Test Controller will be ready to run on your robot.
\section{Usage}

To run the ROS Test Controller, ssh into your robot, navigate to your /ROSTestController directory and run \texttt{npm start}. At bootup, the controller will advise that the export server has initialized (typically on port 8080) and that the main controller is running on port 3000. Leave this SSH session running while you use the Test Controller. To use the main controller, enter the robot's IP in a browser window and port 3000.

For example:

\texttt{192.168.0.0.1:3000} would run the controller for a robot with an IP of 192.168.0.0.1.

At this time, the controller will check that roscore is running and you will be taken to the "Run Test" screen.
\subsection{Run a Test}

To run a test, enter the name of the test you wish to run in the "Test Name" field and click "Initialize". Should initialization be successful, you can then enter the name of the test participant and click "Record Bag". Once the test is concluded, click "Stop Bag". While the test is running, a stopwatch will run to advise how long the bag has been recording.

If you wish to change the test that is initialized, simply enter the new test name and click the "Initialize" button again. Similarly, if you wish to record another test, change the test participant name (if needed) and click 'Record Bag' again.

\subsection{Create a test}
To create a test, click "Create Test" on the top bar of the test controller. Give the test a unique name and list any launch files that are necessary to initialize your robot. List launch files using <package name> <filename> (the same format as is used by roslaunch). If you have multiple launch files, separate each package/launch file pair by commas. 

For example, \texttt{controller\_test turtleSim.launch, controller\_test turtleControls.launch} will specify both the turtleSim.launch and turtleControls.launch files for the test being created.

Once the launch files are specified, the 'Test .launch' button can test that all included launch files will successfully run (both on your robot, as well as with the test controller). If the test fails, any error messages will appear on the ssh console that is running the Test Controller. 

After the launch files have been specified, you can add the topics that you wish to be recorded with your test. Similar to the launch files, separate each topic by a comma. Clicking "Estimate Bag Size" will collect a small bag file using the topics you specified and return an estimated Kb/min for each bag recorded in your test. Note, this feature takes approximately 15 seconds to run, during which any modifications to the topics list will not be included in the estimate. 

Once you are satisfied with the settings for your test, click "Submit Test" and the system will create the setup for your new test. 

\subsection{Export Test Results}
To download any bags that have been recorded under a test, enter the robots IP and the port number specified at boot-up (typically 8080) into your browser of choice. This will display the top level directory for all the tests stored on the robot. Select the name of your test and there will be a directory for each test participant for your tests. Names \textit{will} be encrypted in order to meet IRB requirements. Select the directory you wish to export, and you will be able to download the test recording you desire.

\section{Known Bugs}

\begin{itemize}
	\item ROS Test Controller cannot be run as Super User

	\item At this time, there is no checking if a test name has already been used. If a test name is reused, the previous test configuration file will be overwritten.

	\item If the Test Controller crashes, the export server (a separate process) will still run in the background. This process can be killed by using \texttt{pa -a} to get the pid of the server (it will be a node process) and \texttt{sudo kill -2 <pid>} to kill this process. Otherwise, the next instance of Test Controller will use a new port and this can cause multiple file server ports to be open on your robot.

	\item On occasion, the initialize button will become unresponsive. Refresh the page and this behavior will be corrected.
\end{itemize}

\section{Future Enhancement Possibilities}

\begin{itemize}
	\item Change the Test Name field to a drop down of current tests to prevent typos for test name confusion from causing "No Test Found" errors when initializing.

	\item There is infrastructure to allow test participant names to be pre-specified when a test is created. If implemented, "Participant Name" field could be a drop down to prevent name typos from causing tests from the same participant to be stored in separate directories.

	\item Move all ROSHandling functionality from Python calls to nodeJS functions. This would allow the Test Controller to be a pure nodeJS solution and possibly simplify the code base.

	\item Refactor ros.js to improve code clarity and eliminate redundancy.

	\item Because a user must log into a robot in order to start ROS Test Controller, the decision was made for the Test Controller to not have a separate authentication system for the Controller. With that said, the addition of a separate authentication system (and heirarchy to support 'test creator' vs 'authorized tester') could improve the security and enhance usability.
\end{itemize}

\section{Contributing}

Submit pull requests if you have an update that improves the functionality of this Controller and an author or the Project Owner will test and approve changes.

\chapter{Recommended Technical Resources}
\minitoc

\chapter{Conclusions and Reflections}
\minitoc
\section{Marie's Reflections}
\subsection{What technical information did you learn?}
Over the course of this project, I gained technical proficiency in three main areas, the Robot Operating System (ROS), nodeJS and finally, general web design. 

From ROS, I gained a basic understanding of the interactions within the Robot Operating system, including how to initialize a robot and record a rosbag. I learned how to create packages using catkin, and how to create and run ROS launch files. In addition, I gained an understanding of how different ROS systems interact with one another, and how to debug ROS output to identify system issues. While I will by no means claim to be even a novice ROS users at this point (as most of my work was not with designing ROS systems, but instead interacting with systems already created); I am now at least able to interact with a robot and do basic debugging. 

For nodeJS, while I had interacted with javascript in the past, it had been almost two years from the last time I had worked with it. It was almost like learning a new language when I first started helping with the core ros.js functionality when we restarted the project mid-way through winter term. At first I was a bit hesitant to dive into node, mostly because I am not a fan of web development. But once the first supplemental functions were created, it was easy to pick up what was necessary for each page call, and build a robust system from there. I was even able to develop callbacks so that the system could more easily give feedback to the user. 

Lastly, web design. During winter term, Aaron and I took Intro to Usability Engineering and the best practices from that class did make it in to the final design. While there are still features and design decisions I would like to add to improve the usability of the Test Controller; the base functionality created is fairly straight forward and clear. From the general aesthetics that Aaron created, I was able to tweak and supplement features to better meet the specifications of our design document and help polish the final product that we delivered to our client. 

\subsection{What non-technical information did you learn?}
The biggest non-technical skills I picked up over the course of this project were mostly related to project management (as I delve into in the next questions), but I also learned a lot about work-life balance. During fall term, even though I felt like a had a good grasp on the scope and requirements of the project, I did not have a good grasp on balancing my stress levels regarding the project needs. Toward the end of fall term this turned into panic attacks when working on the Technical Review and quite a lot of energy spent panicking about both assignments and the success of the project as a whole. It took until Winter term for me to calm down and be able to approach each task with a clear head, and understand that my desires for perfection in each and every portion of this project was not only unreasonable to ask of myself, but also of my teammates. Could more have been done? Perhaps, but our client seems satisfied with our results and that is what is most important. 

\subsection{What have you learned about project work?}
My biggest areas of growth in this project relate to both project work and project management. as previously mentioned, anxiety and work/life balance were major struggles for me at the beginning of the school year. Also difficult for me was finding a compromise between my expectations and the expectations of my teammates. I have always been a perfectionist when it comes to my school work and being able to let go and accept that Aaron and Hadi's contributions were just as good, even if they weren't in my particular style, was a challenge. This led to a bit of stress when documentation was being created and I felt the need to go back and edit so a perfectly fine section of information in  order to meet my standards.

I also learned having a team lead who has a good grasp of each element of the project and the end goal is critical. On the flip side, I also learned that having the entire team have this level of understanding of a project is unrealistic. When you have three individuals working on a team, you'll likely have six different ideas of what the end product will look like. Having one person be in charge of this final goal not only makes sure how each element interacts is understood and taken care of, it also helps prevent a project from spiraling out of control if team members have different ideas how the project should go. 
\subsection{What have you learned about project management?}
\subsection{What have you learned about working in teams?}
\subsection{If you could do it all over, what would you do differently?}
\section{Aaron's Reflections}
\subsection{What technical information did you learn?}
\subsection{What non-technical information did you learn?}
\subsection{What have you learned about project work?}
\subsection{What have you learned about project management?}
\subsection{What have you learned about working in teams?}
\subsection{If you could do it all over, what would you do differently?}
\section{Hadi's Reflections}
\subsection{What technical information did you learn?}
\subsection{What non-technical information did you learn?}
\subsection{What have you learned about project work?}
\subsection{What have you learned about project management?}
\subsection{What have you learned about working in teams?}
\subsection{If you could do it all over, what would you do differently?}

\appendix
\minitoc
\section{Essential Code Listings}


\end{document}
