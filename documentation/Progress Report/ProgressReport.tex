\documentclass[onecolumn, draftclsnofoot,10pt, compsoc]{IEEEtran}

%slightly modified from stackoverflow @ https://tex.stackexchange.com/questions/200437/numbering-sections-subsections-etc-manually
%code block below allows for references to function as a section instead of a chapter
\makeatletter
\renewenvironment{thebibliography}[1]
{\subsection{References}
	\@mkboth{\MakeUppercase\bibname}{\MakeUppercase\bibname}%
	\list{\@biblabel{\@arabic\c@enumiv}}%
	{\settowidth\labelwidth{\@biblabel{#1}}%
		\leftmargin\labelwidth
		\advance\leftmargin\labelsep
		\@openbib@code
		\usecounter{enumiv}%
		\let\p@enumiv\@empty
		\renewcommand\theenumiv{\@arabic\c@enumiv}}%
%	\sloppy
	\clubpenalty4000
	\@clubpenalty \clubpenalty
	\widowpenalty4000%
	\sfcode`\.\@m}
{\def\@noitemerr
	{\@latex@warning{Empty `thebibliography' environment}}%
	\endlist}
\makeatother

\usepackage{graphicx}
\usepackage{url}
\usepackage{setspace}
\makeindex
\usepackage{geometry}

\geometry{textheight=9.5in, textwidth=7in}

% 1. Fill in these details
\def \CapstoneTeamName{		Wheelchair Data Collection Team}
\def \CapstoneTeamNumber{		4}
\def \GroupMemberOne{			Marie Bomber}
\def \GroupMemberTwo{			Aaron Leondar}
\def \GroupMemberThree{			Hadi Rahal-Arabi}
\def \CapstoneProjectName{			Robotic Wheelchair Data Collection and Analysis}
\def \CapstoneSponsorCompany{	Oregon State University}
\def \CapstoneSponsorPerson{	Matthew William Shuman	}

% 2. Uncomment the appropriate line below so that the document type works
\def \DocType{	%Problem Statement
				%Requirements Document
				%Technology Review
				%Design Document
					Progress Report
				}
\bibliographystyle{ieeetran}	
\newcommand{\NameSigPair}[1]{\par
\makebox[2.75in][r]{#1} \hfil 	\makebox[3.25in]{\makebox[2.25in]{\hrulefill} \hfill		\makebox[.75in]{\hrulefill}}
\par\vspace{-12pt} \textit{\tiny\noindent
\makebox[2.75in]{} \hfil		\makebox[3.25in]{\makebox[2.25in][r]{Signature} \hfill	\makebox[.75in][r]{Date}}}}
% 3. If the document is not to be signed, uncomment the RENEWcommand below
%\renewcommand{\NameSigPair}[1]{#1}

%%%%%%%%%%%%%%%%%%%%%%%%%%%%%%%%%%%%%%%
\begin{document}
\begin{titlepage}
    \pagenumbering{gobble}
    \begin{singlespace}
        \hfill 
        % 4. If you have a logo, use this includegraphics command to put it on the coversheet.
        %\includegraphics[height=4cm]{CompanyLogo}   
        \par\vspace{.2in}
        \centering
        \scshape{
            \huge CS Capstone \DocType \par
            {\large 27 October 2017}\par
            \vspace{.5in}
            \textbf{\Huge\CapstoneProjectName}\par
            \vfill
            {\large Prepared for}\par
            \Huge \CapstoneSponsorCompany\par
            \vspace{5pt}
            {\Large\NameSigPair{\CapstoneSponsorPerson}\par}
            {\large Prepared by }\par
            Group\CapstoneTeamNumber\par
            % 5. comment out the line below this one if you do not wish to name your team
            \CapstoneTeamName\par 
            \vspace{5pt}
            {\Large
                \NameSigPair{\GroupMemberOne}\par
                \NameSigPair{\GroupMemberTwo}\par
                \NameSigPair{\GroupMemberThree}\par
            }
            \vspace{20pt}
				\begin{abstract}
This document serves to provide the overall design of the entire project. It includes the communication between the wheelchair computer and the personal computer, the methods used to store and present the data, and the interface that brings the entire process together. All of the technologies talked about in this document are the technologies that were chosen by our team to use with our project because they were determined to be the most advantageous to the goals of our project. This document does not contain the alternative technologies that our team looked at and researched, but ultimately did not choose.
				\end{abstract} 
        }   
    \end{singlespace}
\end{titlepage}
\newpage
\pagenumbering{arabic}
\tableofcontents
%\listoffigures
%\listoftables
\clearpage


\section{Introduction}
%Project purposes and goals go here


\section{Project Status}
%Be sure to mention work on Wednesday

\section{Project Problems and Solutions}

\section{Weekly Summary}
%Intro blurb needed

\subsection{Week 1}
Marie: Week one was focused on preparing for the term. Specifically, narrowing down my project selections and focusing where I would like to dedicate my energy. I also updated my resume and wrote a professional biography. 


\subsection{Week 2}
Marie: After project assignments, week two was focused on reaching out to our client and starting to gather details about the project scope. Our first meeting was that Wednesday (October 4th). At this point, we started looking into ROS and outlining development styles and organization tools. We also began work on the Project Proposal Document.


\subsection{Week 3}
Marie: Once the individual rough draft of the problem statement was submitted and reviewed, we got in touch with our TA, Junki and met with our client for the second time. We were able to narrow the scope of the assignment to accumulation and storage of the ROSbag data from research participant trials, as well as an interface to streamline the test process.

\subsection{Week 4}
Marie: In week four we shifted technologies slightly and began to use a waffle.io board to track tasks. This was also the first opportunity we had to see the Perimobile wheelchair that research participants will be testing with. While we did not have the opportunity to try to connect with the wheelchair, we were able to start planning what layers the application would need and assign what each team member would be responsible for. 


\subsection{Week 5}
Marie: Week five was focused on outlining the Requirements document. At this stage each of us began narrowing our focus on our respective systems, and I started managing the waffle board to delegate tasks.

\subsection{Week 6}
Marie: This week was also predominately focused on the Requirements document. We met with the client on Thursday and outlined a timeline for when specific elements of the project would need to be done. We also began planning time during finals week to interact with the wheelchair and test connections as access to the chair needs to be shared with other projects.

\subsection{Week 7}
Marie: After finishing the requirements, we focused on specific technologies that each component of the system would use. At this time, I was weighing the client needs of maintainability and security when storing trial data.

\subsection{Week 8}
Marie: I was not able to work on project responsibilities very much during week eight due to health issues, but I was able to start defining what technologies I would focus on for the review.

\subsection{Week 9}
Marie: In week nine, I discovered several possible python solutions to my database concerns and was able to decide on our approach to storing user trials. Most of my time this week focused on researching python solutions to different levels of the application so we could have as much of an end-to-end python solution as possible. 

\subsection{Week 10}
Marie: In the final week of the term, the team was able to spend time mapping the design of the system and what interactions each component would have with one another. While we discovered there was a bit of confusion on individual elements, between a couple team meetings and a touch-base with the client we were able to finalize what approach we will take. During this time we were also working on the design document and progress report. 

\subsection{Week 11}
%So we can talk about what we are doing this week, may remove later so don't worry about it

\section{Retrospective}

\begin{tabular*}{\linewidth}{@{\extracolsep{\fill}}| p{0.3\linewidth}| p{0.3\linewidth}| p{0.3\linewidth}|@{}}

	\centering Positives & \centering Deltas & \centering Actions \tabularnewline 
\hline 
\end{tabular*}


\bibliography{ProjectBib}

\end{document}