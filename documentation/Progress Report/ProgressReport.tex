\documentclass[onecolumn, draftclsnofoot,10pt, compsoc]{IEEEtran}

%slightly modified from stackoverflow @ https://tex.stackexchange.com/questions/200437/numbering-sections-subsections-etc-manually
%code block below allows for references to function as a section instead of a chapter
\makeatletter
\renewenvironment{thebibliography}[1]
{\subsection{References}
	\@mkboth{\MakeUppercase\bibname}{\MakeUppercase\bibname}%
	\list{\@biblabel{\@arabic\c@enumiv}}%
	{\settowidth\labelwidth{\@biblabel{#1}}%
		\leftmargin\labelwidth
		\advance\leftmargin\labelsep
		\@openbib@code
		\usecounter{enumiv}%
		\let\p@enumiv\@empty
		\renewcommand\theenumiv{\@arabic\c@enumiv}}%
%	\sloppy
	\clubpenalty4000
	\@clubpenalty \clubpenalty
	\widowpenalty4000%
	\sfcode`\.\@m}
{\def\@noitemerr
	{\@latex@warning{Empty `thebibliography' environment}}%
	\endlist}
\makeatother

\usepackage{graphicx}
\usepackage{url}
\usepackage{setspace}
\makeindex
\usepackage{geometry}

\geometry{textheight=9.5in, textwidth=7in}

% 1. Fill in these details
\def \CapstoneTeamName{		Wheelchair Data Collection Team}
\def \CapstoneTeamNumber{		4}
\def \GroupMemberOne{			Marie Bomber}
\def \GroupMemberTwo{			Aaron Leondar}
\def \GroupMemberThree{			Hadi Rahal-Arabi}
\def \CapstoneProjectName{			Robotic Wheelchair Data Collection and Analysis}
\def \CapstoneSponsorCompany{	Oregon State University}
\def \CapstoneSponsorPerson{	Matthew William Shuman	}

% 2. Uncomment the appropriate line below so that the document type works
\def \DocType{	%Problem Statement
				%Requirements Document
				%Technology Review
				%Design Document
					Progress Report
				}
\bibliographystyle{ieeetran}	
\newcommand{\NameSigPair}[1]{\par
\makebox[2.75in][r]{#1} \hfil 	\makebox[3.25in]{\makebox[2.25in]{\hrulefill} \hfill		\makebox[.75in]{\hrulefill}}
\par\vspace{-12pt} \textit{\tiny\noindent
\makebox[2.75in]{} \hfil		\makebox[3.25in]{\makebox[2.25in][r]{Signature} \hfill	\makebox[.75in][r]{Date}}}}
% 3. If the document is not to be signed, uncomment the RENEWcommand below
%\renewcommand{\NameSigPair}[1]{#1}

%%%%%%%%%%%%%%%%%%%%%%%%%%%%%%%%%%%%%%%
\begin{document}
\begin{titlepage}
    \pagenumbering{gobble}
    \begin{singlespace}
        \hfill 
        % 4. If you have a logo, use this includegraphics command to put it on the coversheet.
        %\includegraphics[height=4cm]{CompanyLogo}   
        \par\vspace{.2in}
        \centering
        \scshape{
            \huge CS Capstone \DocType \par
            {\large 27 October 2017}\par
            \vspace{.5in}
            \textbf{\Huge\CapstoneProjectName}\par
            \vfill
            {\large Prepared for}\par
            \Huge \CapstoneSponsorCompany\par
            \vspace{5pt}
            {\Large\NameSigPair{\CapstoneSponsorPerson}\par}
            {\large Prepared by }\par
            Group\CapstoneTeamNumber\par
            % 5. comment out the line below this one if you do not wish to name your team
            \CapstoneTeamName\par 
            \vspace{5pt}
            {\Large
                \NameSigPair{\GroupMemberOne}\par
                \NameSigPair{\GroupMemberTwo}\par
                \NameSigPair{\GroupMemberThree}\par
            }
            \vspace{20pt}
				\begin{abstract}
This document outlines the progress that has been made during the Fall 2017 term on the Project Chiron Interface. It will outline the purpose of the project and the current project status, as well as a summary of the progress made on a week by week basis. This document will also outline successes and challenges encountered over the scope of the Fall 2017 term and include plans to improve performance with Winter 2018 and Spring 2018 terms.
				\end{abstract} 
        }   
    \end{singlespace}
\end{titlepage}
\newpage
\pagenumbering{arabic}
\tableofcontents
%\listoffigures
%\listoftables
\clearpage


\section{Introduction}
%Project purposes and goals go here
\subsection{Background}
Project Chiron is a research project with the goal of designing a kit that can adapt to a motorized wheelchair (specifically a perimobil wheelchair) that will assist people with extremely limited mobility to interact with the world. For this project to become a reality, extensive testing must be done to assess how well someone who is using the Project Chiron interface is using the wheelchair. In order to assess wheelchair competency, typically a wheelchair user would need to schedule an appointment with an Occupational Therapist who is trained to perform a Power-Mobility Indoor Driving
Assessment (PIDA). Such occupational therapists can often be hundreds of miles away (in this case, the closest is in Portland). To that end, our client Professor Matthew Shuman is researching using POM-DP analysis to assess the competency of a wheelchair user using tools built into Project Chiron, specifically, LIDAR sensors tracking wheel movement. 
\subsection{Project Goals}
While Project Chiron is already built and is ready for assessment, gathering research data from the chair is difficult. The chair is running the Robot Operating System (ROS) and can record sensor data, but it takes extensive training to connect to the wheelchair, begin a user trial, and access that information at a later time. This learning curve limits the amount of research that can be done, whereas a easier, point and click interface could allow a paid student worker with relatively low skill to run research tests while the primary researcher is working on other things.

To that end, we are creating a user interface that can signal the beginning of a user test, collect and store only the relevant research data (to limit the size) and then retrieve it at a later time. As the user tests need to be completed in time for our client to conduct his research and publish for two specific conferences, the interface must be completed by mid February. After the interface is completed, the project will shift focus to documenting the hardware that was added to the wheelchair, so that future developers can recreate the system.

Beyond the scope of Project Chiron, needing a skilled worker to run tests is a common limiting factor to how much testing is done in the Robotics Lab. If our interface succeeds in creating an easy, click-to-run interface using ROS, it may be expanded to be used in future robotics projects at OSU. While developing the interface for Project Chiron, we will need to ensure that the system is not too specialized, to aid future robotics research. 

\section{Project Status}
The current status of our project will be a lot more ingrained come Wednesday, December 6th. As of this point, however, we have outlined a schedule for all of winter term in order to ensure that our project reaches certain milestones at specific points in time. The wheelchair we will be using has already been set up, and we will get roughly 4 hours on December 6th in order to test it ourselves to be ready for winter term. We have developed a bare bones model of what the UI that research participant will use, which includes setting the name of the participant, and starting and stopping the recording. Modifications will be made to that model at some point in December. For data storage, we have decided to use NoSQL in order to store the data that is received. The type of NoSQL will be dependent on the quality of internet connection we are able to get in the basement of Dearborn, which will be determined on December 6th as well.

\section{Project Problems and Solutions}
The primary problem we had this quarter seemed to be coordinating schedules. With Marie only available on campus 3 days a week and each of us having jobs outside of classroom requirements, most project organization needed to be done via slack. While this is good for short term delegation and coordination, later in the term it became apparent that there was some misunderstandings that had occurred related to key portions of the project. In future terms, all three members of the team will be available on campus, and Marie has assumed a project manager-type role to reduce the chance of misunderstandings in the future. 

\section{Weekly Summary}
A week-by-week summary of what was accomplished during that particular week, split up by group member. A lot of our weekly summaries are very similar, especially for the first few weeks, because we had not offcially split the project responsibilities until around halfway through the term.

\subsection{Week 1}
Marie: Week one was focused on preparing for the term. Specifically, narrowing down my project selections and focusing where I would like to dedicate my energy. I also updated my resume and wrote a professional biography.\par

Aaron: Wrote my professional biography and looked at each of the projects that were available to choose from. Started eliminating projects that I had no interest in, eventually narrowing my main interests to projects that focused on Robotics.\par

\subsection{Week 2}
Marie: After project assignments, week two was focused on reaching out to our client and starting to gather details about the project scope. Our first meeting was that Wednesday (October 4th). At this point, we started looking into ROS and outlining development styles and organization tools. We also began work on the Project Proposal Document.\par

Aaron: Starting off week two, I received the project I had been assigned, which was Robotic Wheelchair Data Collection and Analysis. During the week, I met with my other group members on Tuesday, October 4th, and then with our client, Matthew Shuman, a day later. We began to work on the problem statement, and working with ROS.\par

\subsection{Week 3}
Marie: Once the individual rough draft of the problem statement was submitted and reviewed, we got in touch with our TA, Junki and met with our client for the second time. We were able to narrow the scope of the assignment to accumulation and storage of the ROSbag data from research participant trials, as well as an interface to streamline the test process.\par

Aaron: We had another meeting with our client and discussed how the data gathered from trials was to be received and stored. We also had our first meeting with our TA Junki, whom we showed our newly created Github repo to. After those two meetings, we had enough of a grasp on the project in order to finish the problem statement and submit it.\par

\subsection{Week 4}
Marie: In week four we shifted technologies slightly and began to use a waffle.io board to track tasks. This was also the first opportunity we had to see the Permobile wheelchair that research participants will be testing with. While we did not have the opportunity to try to connect with the wheelchair, we were able to start planning what layers the application would need and assign what each team member would be responsible for.\par

Aaron: We introduced a different approach to delegate tasks to each group member in the form of a Waffle.io board. We also got a chance to see firsthand the Permobil wheelchair we will be working with, although we did not get a chance to actually test it out at that point. After getting a feel for what would be required of us for the project, we started dividing up main resposibilites.\par

\subsection{Week 5}
Marie: Week five was focused on outlining the Requirements document. At this stage each of us began narrowing our focus on our respective systems, and I started managing the waffle board to delegate tasks.\par

Aaron: The main work that was done during Week 5 was using the information gathered from the meeting the previous week to write up the rough draft of our group's requirements document. I started taking on the Robot Communication aspect of the project as my own section of the project, meaning that I would mainly the ROS aspects of the project.\par

\subsection{Week 6}
Marie: This week was also predominately focused on the Requirements document. We met with the client on Thursday and outlined a timeline for when specific elements of the project would need to be done. We also began planning time during finals week to interact with the wheelchair and test connections as access to the chair needs to be shared with other projects.\par

Aaron: The main work that was done for week 6 was fairly the same for week 5, as it consisted of revising the Requirements document. We both revised it to make the writing flow smoother, and also revised it to better reflect the information we gained from the client meeting we had on Nov. 2nd, as well as a meeting we had with Kirsten on Nov. 1st. During that meeting, we fleshed out the requirements more, using the our rough draft as reference, and also set a rough timeline for when we planned to have certain parts of the project done. The most important thing we planned was to get actual hands-on time with the wheelchair during finals week.\par

\subsection{Week 7}
Marie: After finishing the requirements, we focused on specific technologies that each component of the system would use. At this time, I was weighing the client needs of maintainability and security when storing trial data.\par

Aaron: For week 7, we kind of split off to work on our own individual parts of the tech review document. For my particular portion of the tech review, I was researching alternatives to ROS to see if other robotics middleware could accomplish the same kind of requirements that was needed for the project.\par

\subsection{Week 8}
Marie: I was not able to work on project responsibilities very much during week eight due to health issues, but I was able to start defining what technologies I would focus on for the review.\par

Aaron: Week 8 was very much the same as week 7 for me. My portion of the Tech Document was finished, with my main topics of focus being the Robotics Middleware that we would decide on, the method of gathering and transferring data, and the network connection between the robot and the computer.\par

\subsection{Week 9}
Marie: In week nine, I discovered several possible python solutions to my database concerns and was able to decide on our approach to storing user trials. Most of my time this week focused on researching python solutions to different levels of the application so we could have as much of an end-to-end python solution as possible.\par

Aaron: Week 9 consisted of me making several changes to my tech review document after receiving a hefty amount of critique the previous week. I was not able to get any other work done this week due to separate obligations.\par

\subsection{Week 10}
Marie: In the final week of the term, the team was able to spend time mapping the design of the system and what interactions each component would have with one another. While we discovered there was a bit of confusion on individual elements, between a couple team meetings and a touch-base with the client we were able to finalize what approach we will take. During this time we were also working on the design document and progress report.\par

Aaron: In week 10, we did some additional work on fleshing out the design, as well as planning for the time we were getting to spend with the wheelchair on December 6th. At the beginning of the week, I was a little bit confused on what exactly I would be doing with ROS, and how set up the wheelchair was to be run currently, but was able to relieve some of that confusion with a couple team meetings and a client meeting. We also assembled our design document as well as our progress report.\par

\subsection{Week 11}
%So we can talk about what we are doing this week, may remove later so don't worry about it

\section{Retrospective}

\begin{tabular*}{\linewidth}{@{\extracolsep{\fill}} p{0.3\linewidth}| p{0.3\linewidth}| p{0.3\linewidth}@{}}

	\centering Positives & \centering Deltas & \centering Actions \tabularnewline 
	\hline 
	Over the past ten weeks our understanding of the project has solidified and we have a plan for development going forward. We were able to organize ourselves and move from three individuals with different understandings of how the project will work to a unified goal. & In the next term, we will need to be a bit more proactive in meeting our deadlines. A lot of components were attempted to be completed early, but as our client needs this interface as soon as possible to begin testing, we cannot wait until March to deliver the interface. & As of last week, we have re-organized the project structure slightly to ensure we can be ready to meet the February goal. 

\end{tabular*}


\bibliography{ProjectBib}

\end{document}