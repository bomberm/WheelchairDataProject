\documentclass[onecolumn, draftclsnofoot,10pt, compsoc]{IEEEtran}
\usepackage{graphicx}
\usepackage{url}
\usepackage{setspace}
\makeindex
\usepackage{geometry}
\geometry{textheight=9.5in, textwidth=7in}

% 1. Fill in these details
\def \CapstoneTeamName{		Project Chiron}
\def \CapstoneTeamNumber{		4}
\def \GroupMemberOne{			Marie Bomber}
\def \GroupMemberTwo{			Aaron Leondar}
\def \GroupMemberThree{			Hadi Rahal-Arabi}
\def \CapstoneProjectName{			Robotic Wheelchair Data Collection and Analysis}
\def \CapstoneSponsorCompany{	Oregon State University}
\def \CapstoneSponsorPerson{	Matthew William Shuman	}

% 2. Uncomment the appropriate line below so that the document type works
\def \DocType{	%Problem Statement
				Requirements Document
				%Technology Review
				%Design Document
				%Progress Report
				}
			
\newcommand{\NameSigPair}[1]{\par
\makebox[2.75in][r]{#1} \hfil 	\makebox[3.25in]{\makebox[2.25in]{\hrulefill} \hfill		\makebox[.75in]{\hrulefill}}
\par\vspace{-12pt} \textit{\tiny\noindent
\makebox[2.75in]{} \hfil		\makebox[3.25in]{\makebox[2.25in][r]{Signature} \hfill	\makebox[.75in][r]{Date}}}}
% 3. If the document is not to be signed, uncomment the RENEWcommand below
%\renewcommand{\NameSigPair}[1]{#1}

%%%%%%%%%%%%%%%%%%%%%%%%%%%%%%%%%%%%%%%
\begin{document}
\begin{titlepage}
    \pagenumbering{gobble}
    \begin{singlespace}
        \hfill 
        % 4. If you have a logo, use this includegraphics command to put it on the coversheet.
        %\includegraphics[height=4cm]{CompanyLogo}   
        \par\vspace{.2in}
        \centering
        \scshape{
            \huge CS Capstone \DocType \par
            {\large 27 October 2017}\par
            \vspace{.5in}
            \textbf{\Huge\CapstoneProjectName}\par
            \vfill
            {\large Prepared for}\par
            \Huge \CapstoneSponsorCompany\par
            \vspace{5pt}
            {\Large\NameSigPair{\CapstoneSponsorPerson}\par}
            {\large Prepared by }\par
            Group\CapstoneTeamNumber\par
            % 5. comment out the line below this one if you do not wish to name your team
            \CapstoneTeamName\par 
            \vspace{5pt}
            {\Large
                \NameSigPair{\GroupMemberOne}\par
                \NameSigPair{\GroupMemberTwo}\par
                \NameSigPair{\GroupMemberThree}\par
            }
            \vspace{20pt}
        }   
    \end{singlespace}
\end{titlepage}
\newpage
\pagenumbering{arabic}
\tableofcontents
% 7. uncomment this (if applicable). Consider adding a page break.
%\listoffigures
%\listoftables
\clearpage

% 8. now you write!
\section{Introduction}
\subsection{Purpose}
\subsection{Scope}
\subsection{Definitions, Acronyms and Abbreviations}
\subsection{References}
\subsection{Overview}
\section{Overall Description}
\subsection{Wheelchair to Application Communication}
The application must connect with the wireless network of the wheelchair and receive ROSbags the robot sends after each trial.
The ROSbags, when stored, must contain the length of time of the trial, the ID of the participant, as well as any relevant sensor information.
They cannot contain more than strictly required sensor data because all ROSbags must fit on an 8GB flash drive.
The ROSbags must be able to be sorted by date/time, and also by the participant's ID.
\subsection{Core Architecture}
\subsubsection{ROSbag and UI interface}
\begin{itemize}
	\item The core system should be able to accept a collected ROSbag and correlate it with input from the user interface to create a data entry.
	\subitem The system should recognize if the user entered is a new testee or previous testee and link the new test to previous information.
	\item When passed a request from the user interface, the core system should pass the query to the main storage medium and return from the query in a format that meets the User Interface needs.
	\item The core system should securely store user logins and accept login requests so the interface only requires to handle pass/no pass scenarios.
\end{itemize}
\subsubsection{Data Storage}
\begin{itemize}
	\item The data entry must be stored in a secure storage space that complies with the Institutional Review Board as defined by the client.
	\subitem Testee names must be separate from a unique ID
	\subitem The reference link between testee names and ID's must be able to be purged at the client's request.
\end{itemize}
\subsection{User Interface}
\subsection{Hardware Documentation}
All Hardware that was not included in the off-the-shelf wheelchair must be documented.
\section{Appendixes}



\end{document}