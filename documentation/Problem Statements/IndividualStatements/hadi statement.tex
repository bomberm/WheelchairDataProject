\documentclass[onecolumn, draftclsnofoot,10pt, compsoc]{IEEEtran}
\usepackage{graphicx}
\usepackage{url}
\usepackage{setspace}

\usepackage{geometry}
\geometry{textheight=9.5in, textwidth=7in}

% 1. Fill in these details
\def \CapstoneTeamName{		The Group Formerly Known as 4}
\def \CapstoneTeamNumber{		4}
\def \GroupMemberOne{			Marie Bomber}
\def \GroupMemberTwo{			Aaron Leondar}
\def \GroupMemberThree{			Hadi Rahal-Arabi}
\def \CapstoneProjectName{			Robotic Wheelchair Data Collection and Analysis}
\def \CapstoneSponsorCompany{	Oregon State University}
\def \CapstoneSponsorPerson{	Matthew William Shuman	}

% 2. Uncomment the appropriate line below so that the document type works
\def \DocType{		Problem Statement
				%Requirements Document
				%Technology Review
				%Design Document
				%Progress Report
				}
			
\newcommand{\NameSigPair}[1]{\par
\makebox[2.75in][r]{#1} \hfil 	\makebox[3.25in]{\makebox[2.25in]{\hrulefill} \hfill		\makebox[.75in]{\hrulefill}}
\par\vspace{-12pt} \textit{\tiny\noindent
\makebox[2.75in]{} \hfil		\makebox[3.25in]{\makebox[2.25in][r]{Signature} \hfill	\makebox[.75in][r]{Date}}}}
% 3. If the document is not to be signed, uncomment the RENEWcommand below
%\renewcommand{\NameSigPair}[1]{#1}

%%%%%%%%%%%%%%%%%%%%%%%%%%%%%%%%%%%%%%%
\begin{document}
\begin{titlepage}
    \pagenumbering{gobble}
    \begin{singlespace}
        \hfill 
        % 4. If you have a logo, use this includegraphics command to put it on the coversheet.
        %\includegraphics[height=4cm]{CompanyLogo}   
        \par\vspace{.2in}
        \centering
        \scshape{
            \huge CS Capstone \DocType \par
            {\large 9 October 2017}\par
            \vspace{.5in}
            \textbf{\Huge\CapstoneProjectName}\par
            \vfill
            {\large Prepared for}\par
            \Huge \CapstoneSponsorCompany\par
            \vspace{5pt}
            {\Large\NameSigPair{\CapstoneSponsorPerson}\par}
            {\large Prepared by }\par
            Group\CapstoneTeamNumber\par
            % 5. comment out the line below this one if you do not wish to name your team
            \CapstoneTeamName\par 
            \vspace{5pt}
            {\Large
                \NameSigPair{\GroupMemberOne}\par
                \NameSigPair{\GroupMemberTwo}\par
                \NameSigPair{\GroupMemberThree}\par
            }
            \vspace{20pt}
        }
        \begin{abstract}
        % 6. Fill in your abstract    
		Project Chiron is an ongoing OSU venture to create an electrical kit that allows people with extreme disabilities to use wheelchairs. Our project aims to create an interface for trial data collection when testing project Chiron on Permobil wheelchairs. To create this interface, we will be employing python and Robot Operating System. We will be extending the data collection interface with a data presentation interface for analyzing the data. As we develop the two interfaces we will be creating documentation for the kit based on our interactions with it.  
        \end{abstract}     
    \end{singlespace}
\end{titlepage}
\newpage
\pagenumbering{arabic}
\tableofcontents
% 7. uncomment this (if applicable). Consider adding a page break.
%\listoffigures
%\listoftables
\clearpage

% 8. now you write!
\section{Problem Definition}
Project Chiron, an OSU venture, has developed a prototype kit for allowing those with extreme disabilities to interact with Permobil wheelchairs. While the software and hardware is existing, Project Chiron lacks an interface for data collection and analysis. Our project aims to allow for simple recording of trial data, as well as organizing the data for analysis. Furthermore, there is little documentation on the custom kit that can be mounted on wheelchairs, so it is necessary to document the applications and usage instructions of the kit as we interact with it.
\\\\
The prototype hardware has several sets of sensors that have the capability of constantly collecting data. It will be necessary to develop an understanding of the value of the data collected from each sensor; the data may need to be truncated, or collected at set intervals.
\\\\
Presenting the data in a usable format for analysis provides an entirely different challenge. While data collection will require some understanding of what each sensor is measuring, data analysis and formatting will require an intimate understanding of the purpose of each measurement, as well as the implications of those measurements.


\section{Proposed Solution}
	Our proposed solution will employ a combination of Python and Robot Operating System (ROS). Developing a knowledge of ROS will be necessary to interface with the sensors on the wheelchair’s electrical kit. When the team has developed a grasp of ROS, a data collection interval will be determined for each possible measurement. We will develop a method for transferring data from the kits to our chosen database. Given network compatibility on the kits, there should be REST endpoints for depositing data into a central location. The data will likely be stored in SQL, or a similarly ubiquitous storage solution for maximum compatibility with future iterations of Chiron.
	\\\\
The requirements for data presentation will be fluid as our understanding of data significance changes. For this reason, it is important that the presentation and analytics development portion of the project employs an agile workflow.
\\\\
Documentation will be an ongoing process throughout the project. We must negotiate a performance metric for sufficient documentation, then integrate the documentation process into our weekly workflow. Since this process happens entirely separately from the other project deliverables, it should be delivered first to free time for unexpected changes in requirements during the final term of the project.


\section{Performance Metrics}
	There are three explicit deliverables of the project (copied from the project description): 
\\
\\1. “Data collection interface to gather trial information”
\\
\\The data collection interface can be considered complete when there is a command line interface for collecting data from each of the sensors in the kit, using whatever method of communication the Chiron kit supports. 
\\
\\
2. “Data presentation interface for analysis of the wheelchair path and actions during the recorded trial”
\\
\\The data presentation layer can be considered complete when the data can be represented in at least 3 visual formats. The data presentation layer must also provide a method of pulling data for analysis in third-party tools.\\
\\
3. “Revision and documentation of electrical prototype kit used on the wheelchair.”
\\
\\Documentation performance metrics must be reviewed with the client before their proposal. A set of articles will be determined and scheduled in advance of their actual writing. The definition of “article” will be decided with the client based on their need. 


\end{document}